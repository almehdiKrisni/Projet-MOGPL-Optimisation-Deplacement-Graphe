
\section{Préliminaires}
\label{sec:preliminaires}

\begin{question}
  En utilisant l'instance de la figure , montrer que les assertions suivantes
  sont vraies.
\end{question}

\begin{assertion}
  Un sous-chemin préfixe d'un chemin d'arrivée au plus tôt peut ne pas
  être un chemin d'arrivée au plus tôt.
\end{assertion}


\begin{reponse}
  Considérons le multigraphe de l'Exemple 1. Les chemins réalisables de $a$ à
  $k$ sont les suivants:

  \begin{equation}
    \begin{align}
      \label{eq:1}
      P_{a \rightarrow k} = \{ & P_1 = ((a,b,1,1), (b,g,3,1), (g,k,6,1)), \\
                               & P_2 = ((a,b,1,1), (b,h,3,1), (h,k,7,1)), \\
                               & P_3 = ((a,b,1,1), (b,g,3,1), (g,k,6,1)), \\
                               & P_4 = ((a,b,2,1), (b,g,3,1), (g,k,6,1)), \\
                               & P_5 = ((a,b,2,1), (b,h,3,1), (h,k,7,1)), \\
                               & P_6 = ((a,b,2,1), (b,g,3,1), (g,k,6,1)), \\
                               & P_7 = ((a,c,4,1), (c,h,6,1), (h,k,7,1)) \}.
    \end{align}
  \end{equation}

  Considérons les dates de fin $fin(P_x)$ pour tout chemin
  $P_x \in P_{a \rightarrow k}$:

  \begin{equation}
    \begin{align}
      \label{eq:2}
      & fin(P_1) = 6 + 1 = 7, \\
      & fin(P_2) = 7 + 1 = 8, \\
      & fin(P_3) = 6 + 1 = 7, \\
      & fin(P_4) = 6 + 1 = 7, \\
      & fin(P_5) = 7 + 1 = 8, \\
      & fin(P_6) = 6 + 1 = 7, \\
      & fin(P_7) = 7 + 1 = 8.
    \end{align}
  \end{equation}

  Les chemins d'arrivée au plus tôt
  $$P_{\text{au plus tôt }(a \rightarrow k)} = \{P \in \mathcal{P}(a,k,[1,10]):
  fin(P) = \min(\{fin(P^{'})): P^{'} \in \mathcal{P}(a,k,[1,10])\})\}$$ sont
  donc les suivants:

  \begin{equation}
    \label{eq:3}
    P_{\text{au plus tôt }(a \rightarrow k)} = \{P_1, P_3, P_4, P_6\}
  \end{equation}

  Considérons le chemin $P_4$, un chemin d'arrivée au plus tôt.
  $P_{4 (a \rightarrow b, \text{ jour 2})} = ((a,b,2,1))$ est un sous-chemin
  préfixe (un sous-chemin partant du sommet de départ) de $P_4$, avec une date
  de fin $fin(P_{4 (a \rightarrow b, \text{ jour 2})}) = 2 + 1 = 3$.

  Cependant, $P_{(a \rightarrow b, \text{ jour 1})} = ((a,b,1,1))$ a une date de
  fin $fin(P_{(a \rightarrow b, \text{ jour 1})}) = 1 + 1 = 2$, et
  $fin(P_{4 (a \rightarrow b, \text{ jour 2})}) > fin(P_{(a \rightarrow b,
    \text{ jour 1})})$ - donc $P_{4 (a \rightarrow b, \text{ jour 2})}$
  n'est pas un chemin d'arrivée au plus tôt.

  Ainsi, un sous-chemin préfixe d'un chemin d'arrivée au plus tôt peut ne pas
  être un chemin d'arrivée au plus tôt.
\end{reponse}

\begin{assertion}
  Un sous-chemin postfixe d'un chemin de départ au plus tard peut ne pas être un
  chemin de départ au plus tard.
\end{assertion}

\begin{assertion}
  Un sous-chemin d'un chemin le plus rapide peut ne pas être un chemin le plus
  rapide.
\end{assertion}

\begin{assertion}
  Un sous-chemin d'un plus court chemin peut ne pas être un plus court chemin.
\end{assertion}

%%% Local Variables:
%%% mode: latex
%%% TeX-master: "../main"
%%% End:

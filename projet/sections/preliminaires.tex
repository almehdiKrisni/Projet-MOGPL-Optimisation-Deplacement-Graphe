
\section{Préliminaires}
\label{sec:preliminaires}

\begin{question}
  En utilisant l'instance de la figure de gauche de l'Exemple 1 (dans l'énoncé)
  ou une autre instance, montrer que les assertions suivantes sont vraies.
\end{question}

Par soucis de simplicité et pour être plus concis, une instance alternative est
proposée. Considérons $G_{1}$, le multigraphe orienté pondéré orienté par le
temps donné par le diagramme de la Figure \ref{fig:G1}:

\begin{figure}[h!]
  \centering
  \begin{tikzpicture}[->,>=stealth',shorten >=1pt,auto,node distance=4cm,
    thick,main node/.style={circle,draw,font=\bfseries}]

    \node[main node] (a) at (0,0) {$a$};
    \node[main node] (b) at (0,-2) {$b$};
    \node[main node] (c) at (0,-4) {$c$};
    \node[main node] (d) at (3,-2) {$d$};
    \node[main node] (e) at (3,-4) {$e$};

    \path
    (a) edge [bend left]  node {$1$} (b)
    (a) edge [bend right] node {$3$} (b)
    (b) edge [bend left]  node {$4$} (c)
    (b) edge [bend right] node {$7$} (c)
    (b) edge node {$2$} (d)
    (b) edge node {$1$} (e)
    (d) edge node {$4$} (e);
  \end{tikzpicture}
  \caption{Diagramme représentant le multigraphe pondéré orienté par le temps $G_{1}$}
  \label{fig:G1}
\end{figure}

Le multigraphe $G_1$ sera utilisé pour justifier les assertions suivantes.

\begin{assertion}
  Un sous-chemin préfixe d'un chemin d'arrivée au plus tôt peut ne pas
  être un chemin d'arrivée au plus tôt.
\end{assertion}


\begin{reponse}
  Considérons l'ensemble de chemins $\mathcal{P}(a,c,[0, \infty])$ dans le
  multigraphe $G_{1}$, qui correspond à l'ensemble de tous les chemins
  réalisables de $a$ à $c$ dans $G_{1}$:

  \begin{equation}
    \begin{align}
      \label{eq:1}
      \mathcal{P}(a,c,[0, \infty]) = \{ & P_1 = ((a,b,1,1), (b,c,4,1)), \\
                                        & P_2 = ((a,b,1,1), (b,c,7,1)), \\
                                        & P_3 = ((a,b,3,1), (b,c,4,1)), \\
                                        & P_4 = ((a,b,3,1), (b,c,7,1)) \}.
    \end{align}
  \end{equation}

  Pour déterminer le(s) chemin(s) d'arrivée au plus tôt de $a$ à $c$ dans le
  graphe $G_{1}$, soit un chemin $P$ tel que
  $\mathrm{fin}(P) = \min(\{\mathrm{fin}(P'): P' \in \mathcal{P}(a,c,[0,
  \infty]\})$, calculons les dates de fin pour tout
  $P$ appartenant à $\mathcal{P}(a,c,[0, \infty])$:

  \begin{equation}
    \begin{align}
      \label{eq:2}
      & \mathrm{fin}(P_1) = 4 + 1 = 5, \\
      & \mathrm{fin}(P_2) = 7 + 1 = 8, \\
      & \mathrm{fin}(P_3) = 4 + 1 = 5, \\
      & \mathrm{fin}(P_4) = 7 + 1 = 8.
    \end{align}
  \end{equation}

  Ainsi, $\min(\{5,8,5,8\}) = 5$ et les chemins d'arrivée au plus tôt sont $P_1$
  et $P_3$.

  $P_{3}' = ((a,b,3,1))$, un chemin de $a$ vers $b$, est un sous-chemin préfixe
  (un sous-chemin partant du sommet de départ) de $P_{3}$. Cependant, il existe
  un chemin $P_{a \rightarrow b} = ((a,b,1,1))$ tel que
  \begin{equation}
    \mathrm{fin}(P_{3}') = 3 + 1 = 4 > \mathrm{fin}(P_{a \rightarrow b}) = 1 + 1
    = 2 \text{,}
  \label{eq:4}
  \end{equation}
  donc $P_{3}'$ n'est pas un chemin d'arrivée au plus tôt de $a$ à $b$.

  Ainsi, un sous-chemin préfixe d'un chemin d'arrivée au plus tôt peut ne pas
  être un chemin d'arrivée au plus tôt.
\end{reponse}

\begin{assertion}
  Un sous-chemin postfixe d'un chemin de départ au plus tard peut ne pas être un
  chemin de départ au plus tard.
\end{assertion}

\begin{reponse}
  Considérons de nouveau l'ensemble de chemins $\mathcal{P}(a,c,[0, \infty])$,
  donné en Équation \ref{eq:1}.

  Un chemin de départ au plus tard de $a$ à $c$, dans le multigraphe $G_1$,
  correspond à un chemin $P$ tel que
  $\mathrm{début}(P) = \max(\{\mathrm{début}(P'): P' \in
  \mathcal{P}(a,c,[0, \infty]\})$. On observe facilement que
  \begin{equation}
    \mathrm{début}(P_1) = \mathrm{début}(P_2) = 1 < \mathrm{début}(P_3) =
    \mathrm{début}(P_4) = 3 \text{,}
  \end{equation}
  donc les chemins de départ au plus tard entre $a$ et $c$ sont $P_3$ et $P_4$.

  $P_{3}'' = ((b,c,4,1))$, un chemin de $b$ vers $c$, est un sous-chemin
  postfixe (un sous-chemin partant du sommet final) de $P_3$. Cependant, il
  existe un chemin $P_{b \rightarrow c} = ((b,c,7,1))$ tel que
  \begin{equation}
    \mathrm{début}(P_{3}'') = 3 < \mathrm{début}(P_{b \rightarrow c}) = 7
    \label{eq:5}
  \end{equation}
  donc $P_{3}''$ n'est pas un chemin de départ au plus tard de $b$ à $c$.

  Ainsi, un sous-chemin postfixe d'un chemin de départ au plus tard peut ne pas
  être un chemin de départ au plus tard.
\end{reponse}

\begin{assertion}
  Un sous-chemin d'un chemin le plus rapide peut ne pas être un chemin le plus
  rapide.
  \label{assert:rapide}
\end{assertion}

\begin{reponse}
  Considérons les chemins de $a$ à $d$ dans $G_{1}$:
  \begin{equation}
    \mathcal{P}(a,d,[0, \infty]) = \{P_5 = ((a,b,1,1),(b,d,2,1),(d,e,4,1))\} \text{.}
    \label{eq:6}
  \end{equation}

  Comme il n'existe qu'un seul chemin réalisable, c'est forcément un chemin le
  plus rapide de $a$ à $d$, un chemin $P$ tel que
  $\mathrm{durée}(P) = \min(\{\mathrm{durée}(P'): P' \in \mathcal{P}(a,d,[0,
  \infty]) \})$.

  $P_{5}' = ((b,d,2,1),(d,e,4,1))$, un chemin de $b$ à $e$, est un sous-chemin
  de $P_5$. Cependant, il existe un chemin $P_{b \rightarrow e} = ((b,e,1,1))$
  de $b$ vers $e$ tel que
  \begin{equation}
    \mathrm{durée}(P_{5}') = (4+1) - 1 = 4 > \mathrm{durée}(P_{b \rightarrow e}) = (1+1) - 1 = 1 \text{,}
    \label{eq:3}
  \end{equation}
  donc $P_{5}'$ n'est pas un chemin le plus rapide de $b$ à $e$.

  Ainsi, un sous-chemin d'un chemin le plus rapide peut ne pas être un chemin le
  plus rapide.
\end{reponse}

\begin{assertion}
  Un sous-chemin d'un plus court chemin peut ne pas être un plus court chemin.
\end{assertion}

\begin{reponse}
  Considérons de nouveau les chemins $\mathcal{P}(a,d,[0, \infty])$ de $a$ vers
  $d$ donnés en Équation \ref{eq:6}.

  De même, comme il n'existe qu'un seul chemin réalisable, $P_5$ est forcément
  le chemin le plus court: un chemin $P$ tel que
  $\mathrm{dist}(P) = \min(\{\mathrm{dist}(P'): P' \in \mathcal{P}(a,d,[0,
  \infty]) \})$.

  Considérons de nouveau les chemins $P_{5}''$ et $P_{b \rightarrow e}$ donnés
  en Assertion \ref{assert:rapide}.
  \begin{equation}
    \mathrm{dist}(P_{5}') = 1 + 1 = 2 > \mathrm{dist}(P_{b \rightarrow e}) = 1 \text{,}
    \label{eq:3}
  \end{equation}
  donc $P_{5}'$ n'est pas un chemin le plus court de $b$ à $e$.

  Ainsi, un sous-chemin d'un chemin le plus court peut ne pas être un chemin le
  plus court.

\end{reponse}

%%% Local Variables:
%%% mode: latex
%%% TeX-master: "../main"
%%% End:

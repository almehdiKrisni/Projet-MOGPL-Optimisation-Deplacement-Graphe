\section{Algorithmes de plus court chemin dans des multigraphes orientés pondérés par le temps}
\label{sec:algos}

Dans cette partie, les réponses aux questions 2 et 3 et du sujet sont données au
fûr et à mesure de la présentation des algorithmes.

Les algorithmes développés pour la question 4 sont mis en annexe à ce rapport.

\begin{question}
  En utilisant comme base la transformation de $G$ à $\tilde{G}$, montrer
  comment calculer de manière efficace les 4 types de chemins minimaux.
\end{question}

\begin{question}
  Calculer la complexité de différents algorithmes proposés.
\end{question}

\begin{question}
  Ecrire un programme qui prend en entrée un multigraphe $G$ sans circuit et
  retourne les 4 types de chemins minimaux correspondants.
\end{question}

Pour obtenir les 4 types de chemins minimaux d'un multigraphe $G$ dans un
fichier txt, on peut faire tourner le programme main. La commande suivante
calcule ces 4 chemins, entre a et c, pour le graphe dans exempleGraphe.txt:
\begin{center}
\verb|$ python3 main.py "Repertoire_Graphes/exempleGraphe.txt" a c|
\end{center}
\subsection{Structure de données pour la représentation de multigraphes}

Pour représenter des multigraphes (et des graphes classiques, également), on
choisit de travailler avec des dictionnaires représentant des listes d'adjacence
(dans ce cas, des listes de succésseurs). Ce choix est efficace vu que l'accès
aux dictionnaires en python est d'une complexité de $\mathcal{O}(1)$, via
l'utilisation fonctions de hashage gérées automatiquement par python.

Pour un multigraphe $G$, les clés du dictionnaire correspondent aux sommets de
$G$, et ses valeurs à la liste des arcs sortants pour chaque sommet. Comme un
arc $(s_1,s_2,dDD,cDT)$ - où $s_1$ correspond au sommet de départ, $s_2$ au
sommet de destination, $dDD$ la date de départ du vol, et $cDT$ la durée du vol
- est forcément associé à un sommet de départ dans le dictionnaire (la clé), on
ne garde que $(s_2,dDD,cDT)$.

\subsection{Transformation multigraphe $\rightarrow$ graphe}
\label{sec:transfo}

Pour adapter des algorithmes standards à des multigraphes orientés pondérés par
le temps (MOPLT), une technique est proposée dans le sujet pour tranformer un
MOPLT en un graphe classique qui prend en compte le temps de manière explicite.

On propose une approche légèrement différente à celle proposée dans le sujet. On
calcule $\tilde{V}_{in}$ et $\tilde{V}_{out}$, et on crée une liste de
d'adjacence. Cependant, on ne donne pas de poids aux arcs: avec une comparaison
en $\mathcal{O}(1)$ on peut déterminer si deux noeuds sont les mêmes (et donc un
poids de 0) où différents (un poids de 1).

Pour le faire, on a implémenté une fonction \verb|transformeGraphe| qui prend en
argument un multigraphe $G$ (représenté sous-forme d'un dictionnaire) et
retourne le graphe $\tilde{G}$ (lui aussi sous forme de dictionnaire). Cette
fois, les clés du dictionnaire représentant $\tilde{G}$ correspondent au tuple
$(s_1,t_1)$ où $s_1 \in G$ et $t_1$ correspond à la valeur de $dDD$ dans un des
arcs sortant de $s_1$. Les valeurs du dictionnaire de $\tilde{G}$ correspondent
à deux listes de sommets $(s_2,t_2)$ et $(s_3,t_3)$ où $(s_2,t_2)$ correspond à
un vol de $s_1$ à $s_2$ le jour $t_2$, et $(s_3,t_3)$ correspond à un vol de
$s_3$ à $s_1$ le jour $t_3$.

Pour que les algorithmes proposés dans les sections suivantes soient plus
efficaces, on choisit également d'ordonner les dates des sommets sortants par
ordre croissante.

Le pseudo-code pour cette fonction est donné en Algorithme \ref{alg:transfo}.


Pour calculer la complexité, on assume que $|V| < |E|$. Ainsi, la complexité
temporelle de l'algorithme 1 est la suivante:

\begin{equation}
\begin{align}
  \mathcal{O}(\text{Algorithme 1}) & = \mathcal{O}((|V| + |E|) + (|V| + (|V|+|E|) \log (|V| + |E|))) \\
                                   & = \mathcal{O}(2(|V| + |E|) + (|V|+|E|)\log(|V|+|E|)) \\
                                   & = \mathcal{O}((|V| + |E|)\log(|V|+|E|))\text{,}
\end{align}
\label{eq:8}
\end{equation}
et la complexité spatiale (la taille de $\tilde{G}$):

\begin{equation}
  \label{eq:9}
  \mathcal{O}(\text{Algorithme 1}) = \mathcal{O}(|V| + |E|)
\end{equation}

\begin{algorithm}[h!]
\caption{tranformeGraphe}\label{alg:transfo}
\KwData{$G$ multigraphe orienté pondéré par le temps, sous forme de dictionnaire}
\KwResult{$\tilde{G}$ graphe orienté, sous forme de dictionnaire}

$\tilde{G} \gets \text{dictionnaire vide}$ \;

\ForEach{$s_i$ in sommets($G$)}{

  \ForEach{$(s_j,dDD,cDT)$ in arcs($G[s_i]$)}{

    $\text{sommet\_courant} \gets (s_i, dDD)$\;
    $\text{sommet\_suivant} \gets (s_j, dDD + cDT)$\;

    % \eIf{$s_i = s_j$}{
    %   $\text{poids} \gets 0$ \;
    % }{
    %   $\text{poids} \gets 1$ \;
    % }

    $\tilde{G}[\text{sommet\_courant}][1] \gets \tilde{G}[\text{sommet\_courant}][1] + (\text{sommet\_suivant})$\;
    $\tilde{G}[\text{sommet\_suivant}][0] \gets \tilde{G}[\text{sommet\_suivant}][0] + (\text{sommet\_courant})$\;

  }
}

\ForEach{$s_i$ in sommets($G$)}{
  \ForEach{$(s_j, dDD_j)$ in sommets($\tilde{G}$)}{
    $\text{sommets\_a\_ajouter} \gets \text{liste vide}$ \;
    \If{$s_i = s_j$}{
      $\text{sommets\_a\_ajouter} \gets \text{sommets\_a\_ajouter} + (s_j,dDD_j)$\;
    }
    $\text{sommets\_a\_ajouter} \gets \text{triage\_croissant}(\text{sommets\_a\_ajouter})$\;
    $\text{idx\_sommet} \gets 1$\;
    \While{$i < len(\text{sommets\_a\_ajouter})$}{
      $\text{sommet\_courant} \gets \text{sommets\_a\_ajouter}[i - 1]$\;
      $\text{sommet\_suivant} \gets \text{sommets\_a\_ajouter}[i]$\;

      $\tilde{G}[\text{sommet\_courant}][1] \gets \tilde{G}[\text{sommet\_courant}][1] + \text{sommet\_suivant}$\;
      $\tilde{G}[\text{sommet\_suivant}][0] \gets \tilde{G}[\text{sommet\_suivant}][0] + \text{sommet\_courant}$\;
      $\text{idx\_sommet} \gets \text{idx\_sommet} + 1$\;
    }

  }
}
\end{algorithm}

\subsection{Problème du chemin d'arrivée au plus tôt}
\label{sec:plus-tot}

Pour le problème du chemin d'arrivée au plus tôt, comme pour les deux problèmes
suivants, on utilise une variante du BFS (Breadth-First Search) - c'est un
algorithme efficace lorsqu'il s'agit de trouver un seul chemin (optimal) d'un
point à un autre dans un graphe. Ce sera la méthode utilisée pour les 3 premiers
problèmes.

Vu que ce n'étais pas précisé, on a décidé de garder uniquement les chemins d'un
point du graphe à un autre. Mais notre algorithme parcourt forcément tous les
points, donc la complexité pour récupérer les chemins les plus court d'un point
à tous les autres serait la même.

L'idée de cet algorithme est de partir du sommet de départ dans $\tilde{G}$ et
de faire tourner BFS.

Pour calculer le chemin d'arrivée au plus tôt, on crée une pile contenant tous
les successeurs du sommet de départ - et on ajoute progressivement les fils de
ces sommets en vérifiant si on est au sommet de destination et en gardant le
meilleur temps (indiqué directement dans le sommet de $\tilde{G}$). En plus de
garder les successeurs, on considère aussi ce qu'on appelle un ``state'': c'est
un tuple contenant le sommet courant, le jour actuel, et le père (le noeud
précédent) de l'état. Pour récupérer le chemin optimal, on n'a qu'à dépiler.

Le pseudo-code pour cet algorithme est donné en Algorithme \ref{alg:plus-tot}.

\begin{algorithm}[h!]
\caption{cheminArriveeAuPlusTot}\label{alg:plus-tot}
\KwData{$\tilde{G}$ graphe orienté, sous forme de dictionnaire. sommetDepart, tuple. sommetArrivee, tuple.}
\KwResult{$res$, liste de sommets de $\tilde{G}$ formant un chemin d'arrivée au plus tôt de sommetDepart à sommetArrivee.}

$\text{pile} \gets \text{liste vide}$\;
$res \gets \text{liste vide}$\;
$\text{state} \gets (\text{sommetDepart}, 1, None)$\;
\eIf{$(\text{sommetDepart},1)$ in sommets($\tilde{G}$)}{
  \ForEach{$(s_i,dDD_i)$ in arcs($\tilde{G}[(\text{sommetDepart},1)]$)}{
    $\text{pile} \gets \text{pile} + (s_i,dDD_i,state)$\;
  }
}{
  \ForEach{$(s_i,dDD_i)$ in sommets($\tilde{G}$) if $s_i = \text{sommetDepart}$}{
    $\text{pile} \gets \text{pile} + (s_i,dDD_i,state)$\;
  }
}
$\text{bestChemin} \gets \text{emtpy tuple}$\;
$\text{bestTime} \gets \text{None}$\;
\While{$\text{pile} \neq \emptyset$}{
  $\text{currentState} \gets \text{pile}[0]$\;
  $\text{pile} \gets \text{pile}[1:]$\;

  \eIf{$\text{currentState} = \text{sommetArrivee}$ and ($\text{bestTime} > \text{currentState}[1]$ or $\text{bestTime} = \text{None}$)}{
      $\text{bestTime} \gets \text{currentState[1]}$\;
      $\text{bestChemin} \gets \text{currentState}$\;
    }{
      \ForEach{$(s_i,dDD_i)$ in sommets($\tilde{G}[(\text{currentState}[0],\text{currentState}[1])]$)}{
        $\text{pile} \gets (s_i,dDD_i, \text{currentState})$\;
        }
    }
}
\While{$\text{bestChemin} \neq \text{None}$}{
  $res \gets (\text{bestChemin}[0], \text{bestChemin}[1])$\;
  $\text{bestChemin} \gets \text{bestChemin}[2]$\;
}
$\text{inverse}(res)$\;
\end{algorithm}


Vu que l'on utilise la représentation $\tilde{G = (V_1,E_1)}$ du multigraphe
$G = (V_2,E_2)$, juste un seul BFS est nécessaire pour trouver le chemin
optimal. Supposons que $|V_2| < |E_2|$: ainsi, tant $|V_1|$ comme $|E_1|$ sont
bornés par $\mathcal{O}(|E_2|)$. La complexité cet algorithme est donc:

\begin{equation}
  \label{eq:10}
  \begin{align}
    \mathcal{O}(\text{Algorithme 2}) & = \mathcal{O}(|V_1| + |E_1|) \\
                                     & = \mathcal{O}(2 |E_{2}|) \\
                                     & = \mathcal{O}(|E_{2}|)\text{.}
  \end{align}
\end{equation}

L'algorithme proposé pour le problème de chemin d'arrivée au plus tôt est donc
en temps linéaire après la transformation du graphe.

\subsection{Problème du chemin de départ au plus tard}
\label{sec:plus-tard}

Pour le problème du chemin de départ au plus tard, on procède d'une manière
similaire à celui de l'arrivée au plus tôt. L'idée est de partir du sommet de
départ dans $\tilde{G}$ et de faire tourner BFS à l'envers. Le temps maximal
trouvé correspond au chemin de départ au plus tard.

Le pseudo-code pour cet algorithme est donné en Algorithme \ref{alg:plus-tard}.

\begin{algorithm}[h!]
\caption{cheminDepartAuPlusTard}\label{alg:plus-tard}
\KwData{$\tilde{G}$ graphe orienté, sous forme de dictionnaire. sommetDepart, tuple. sommetArrivee, tuple.}
\KwResult{$res$, liste de sommets de $\tilde{G}$ formant un chemin de départ au plus tard de sommetDepart à sommetArrivee.}

$\text{pile} \gets \text{liste vide}$\;
$res \gets \text{liste vide}$\;
$\text{state} \gets (\text{sommetDepart}, 1, None)$\;
\eIf{$(\text{sommetDepart},1)$ in sommets($\tilde{G}$)}{
  \ForEach{$(s_i,dDD_i)$ in arcs($\tilde{G}[(\text{sommetDepart},1)]$)}{
    $\text{pile} \gets \text{pile} + (s_i,dDD_i,state)$\;
  }
}{
  \ForEach{$(s_i,dDD_i)$ in sommets($\tilde{G}$) if $s_i = \text{sommetDepart}$}{
    $\text{pile} \gets \text{pile} + (s_i,dDD_i,state)$\;
  }
}
$\text{bestChemin} \gets \text{emtpy tuple}$\;
$\text{lastDeparture} \gets \text{None}$\;
\While{$\text{pile} \neq \emptyset$}{
  $\text{currentState} \gets \text{pile}[0]$\;
  $\text{pile} \gets \text{pile}[1:]$\;

  \eIf{$\text{currentState} = \text{sommetArrivee}$}{
      \While{$\text{currentState}[0] \neq \text{sommetDepart}$}{
        $\text{currentState} \gets \text{currentState}[2]$\;
      }
      $\text{startTime} \gets \text{currentState}[1]$\;

      \If{$\text{lastDeparture} < \text{startTime}[1]$ or $\text{lastDeparture} = \text{None}$}{
        $\text{lastDeparture} \gets \text{startTime}$\;
        $\text{bestChemin} \gets \text{currentState}$\;
      }
    }{
      \ForEach{$(s_i,dDD_i)$ in sommets($\tilde{G}[(\text{currentState}[0],\text{currentState}[1])]$)}{
        $\text{pile} \gets (s_i,dDD_i, \text{currentState})$\;
      }
    }
}
\While{$\text{bestChemin} \neq \text{None}$}{
  $res \gets (\text{bestChemin}[0], \text{bestChemin}[1])$\;
  $\text{bestChemin} \gets \text{bestChemin}[2]$\;
}
$\text{inverse}(res)$\;
\end{algorithm}

Cet algorithme, comme le premier, est également en $\mathcal{O}(|E_2|)$.

\subsection{Problème du chemin le plus rapide}
\label{sec:plus-rapide}

Pour le problème du chemin le plus rapide, on procède encore une fois de manière
similaire aux deux derniers algorithmes. Cette fois, on considère la différence
entre le temps du sommet d'arrivée et le temps du sommet de départ. Une
différence est que l'on utilise une fonction \verb|fatherState()| qui récupère
le state précédent dans le chemin actuel.

Le pseudo-code pour cet algorithme est donné en Algorithme
\ref{alg:plus-rapide}.

\begin{algorithm}[h!]
\caption{CheminPlusRapide}\label{alg:plus-rapide}
\KwData{$\tilde{G}$ graphe orienté, sous forme de dictionnaire. sommetDepart, tuple. sommetArrivee, tuple.}
\KwResult{$res$, liste de sommets de $\tilde{G}$ formant un chemin le plus rapide de sommetDepart à sommetArrivee.}

$\text{pile} \gets \text{liste vide}$\;
$res \gets \text{liste vide}$\;
$\text{state} \gets (\text{sommetDepart}, 1, None)$\;
\eIf{$(\text{sommetDepart},1)$ in sommets($\tilde{G}$)}{
  \ForEach{$(s_i,dDD_i)$ in arcs($\tilde{G}[(\text{sommetDepart},1)]$)}{
    $\text{pile} \gets \text{pile} + (s_i,dDD_i,state)$\;
  }
}{
  \ForEach{$(s_i,dDD_i)$ in sommets($\tilde{G}$) if $s_i = \text{sommetDepart}$}{
    $\text{pile} \gets \text{pile} + (s_i,dDD_i,state)$\;
  }
}
$\text{bestChemin} \gets \text{emtpy tuple}$\;
$\text{shortestTime} \gets \text{None}$\;
\While{$\text{pile} \neq \emptyset$}{
  $\text{currentState} \gets \text{pile}[0]$\;
  $\text{pile} \gets \text{pile}[1:]$\;

  \eIf{$\text{currentState} = \text{sommetArrivee}$}{
    $\text{timeSpent} \gets currentState[1] - fatherState(currentState)[1]$\;
    \If{$\text{shortestTime} > \text{timeSpent}$ or $\text{shortestTime} = \text{None}$}{
      $\text{shortestTime} \gets \text{timeSpent}$\;
      $\text{bestChemin} \gets \text{currentState}$\;
    }
    }{
      \ForEach{$(s_i,dDD_i)$ in sommets($\tilde{G}[(\text{currentState}[0],\text{currentState}[1])]$)}{
        $\text{pile} \gets (s_i,dDD_i, \text{currentState})$\;
        }
    }
}
\While{$\text{bestChemin} \neq \text{None}$}{
  $res \gets (\text{bestChemin}[0], \text{bestChemin}[1])$\;
  $\text{bestChemin} \gets \text{bestChemin}[2]$\;
}
$\text{inverse}(res)$\;
\end{algorithm}

Cet algorithme a une complexité (au pire des cas) de
$\mathcal{O}(|E_2| \log(|E_2|))$, ce qui pourrait être amélioré sans la fonction
\verb|fatherState|. Comme cette fonction n'est que appellée lorsqu'un chemin est
réalisable est trouvé, dans la plupart des cas la complexité se rapproche plus
de $\mathcal{O}(|E_2|)$.

\subsection{Problème du plus court chemin}
\label{sec:plus-court}

Pour le problème du plus court chemin, c'est plus compliqué. On aurait voulu
implémanter une méthode de type Dijkstra, pour laquelle on aurait obtenu une
complexité de $\mathcal{O}(|E_2| \log(|E_2|))$, mais on a pas eu le temps pour
adapter notre structure de données pour fonctionner de cette facon.

Ainsi, notre algorithme possède une compléxité de $\mathcal{O}(|E_2|^{2})$, ce
qui provient de l'utilisation de deux boucles while imbriquées.

Le pseudo-code pour cet algorithme est donné en Algorithme \ref{alg:plus-court}

\begin{algorithm}[h!]
\caption{cheminPlusCourt}\label{alg:plus-court}
\KwData{$\tilde{G}$ graphe orienté, sous forme de dictionnaire. sommetDepart, tuple. sommetArrivee, tuple.}
\KwResult{$res$, liste de sommets de $\tilde{G}$ formant un chemin le plus court de sommetDepart à sommetArrivee.}

$\text{pile} \gets \text{liste vide}$\;
$res \gets \text{liste vide}$\;
$\text{state} \gets (\text{sommetDepart}, 1, None)$\;
\eIf{$(\text{sommetDepart},1)$ in sommets($\tilde{G}$)}{
  \ForEach{$(s_i,dDD_i)$ in arcs($\tilde{G}[(\text{sommetDepart},1)]$)}{
    $\text{pile} \gets \text{pile} + (s_i,dDD_i,state)$\;
  }
}{
  \ForEach{$(s_i,dDD_i)$ in sommets($\tilde{G}$) if $s_i = \text{sommetDepart}$}{
    $\text{pile} \gets \text{pile} + (s_i,dDD_i,state)$\;
  }
}
$\text{bestChemin} \gets \text{emtpy tuple}$\;
$\text{fewestMoves} \gets \text{None}$\;
\While{$\text{pile} \neq \emptyset$}{
  $\text{currentState} \gets \text{pile}[0]$\;
  $\text{pile} \gets \text{pile}[1:]$\;

  \eIf{$\text{currentState} = \text{sommetArrivee}$}{
    $\text{differentPlaces} \gets \text{liste vide}$\;
    $\text{tempState} \gets currentState$\;
    \While{$\text{tempState} \neq \emptyset$}{
      $\text{differentPlaces} \gets \text{differentPlaces} + \text{tempState}[0]$\;
      $\text{tempState} = \text{tempState}[2]$\;
    }
    $\text{numberMoves} = len(differentPlaces)$\;
    \If{$\text{fewestMoves} < \text{numberMoves}$ or $\text{fewestMoves} = \text{None}$}{
      $\text{bestTime} \gets \text{currentState[1]}$\;
      $\text{bestChemin} \gets \text{currentState}$\;
      }
    }{
      \ForEach{$(s_i,dDD_i)$ in sommets($\tilde{G}[(\text{currentState}[0],\text{currentState}[1])]$)}{
        $\text{pile} \gets (s_i,dDD_i, \text{currentState})$\;
        }
    }
}
\While{$\text{bestChemin} \neq \text{None}$}{
  $res \gets (\text{bestChemin}[0], \text{bestChemin}[1])$\;
  $\text{bestChemin} \gets \text{bestChemin}[2]$\;
}
$\text{inverse}(res)$\;
\end{algorithm}

\subsection{Problème du plus court chemin en Programmation Linéaire}
\label{sec:plus-court-pl}

\begin{question}
  Proposer un modélisation du problème de plus court chemin par programmation
  linéaire et implémanter une méthode de recherche de chemin de Type IV en
  faisant appel à Gurobi.
\end{question}

Pour résoudre le problème de plus court chemin par programmation linéaire, on
doit d'abord choisir une représentation pour le graphe qui soit naturelle à
utiliser dans un programme linéaire.

On commence avec le graphe $\tilde{G}$. On représente le graphe par une matrice
d'adjacence $C$ avec les poids des arcs entre les différents sommets.

Les variables de décision $X$ correspondent à une autre matrice, binaire, qui
indique si on considère où non l'arc à la position (i,j) dans le chemin allant
du sommet de départ au sommet de destination.

On introduit les contraintes suivantes:
\begin{enumerate}
\item Il n'y a qu'une seule et unique arrête sortante depuis le sommet de
  départ.
\item Il n'y a qu'une seule et unique arrête entrante sur le sommet d'arrivée.
\item Pour chaque sommet (sauf celui de départ et d'arrivée), s'il existe un arc
  entrant vers le sommet il faut forcément un arc sortant.
\end{enumerate}

La fonction objectif correspond à minimiser: $\sum \sum c_{i,j} \times x_{i,j}$,
pour $c_{i,j} \in C$ et $x_{i,j} \in X$.

La méthode \verb|optPlusCourtChemin()|, qui prend en argument un graphe modifié
par la méthode \verb|transformeGrapheOptimisation()|, calcule un plus court
chemin en résolvant un programme linéaire.

\subsection{Algorithmes sans transformation de graphe}
\label{sec:algos-sans-transfo}

Nous avons aussi développé des algorithmes qui n'utilisent pas la transformation
de graphe. Ils correspondent aux versions V2 dans le fichier
\verb|algorithmesChemin.py|. Ils sont bien commentés donc le rapport ne les
abordera pas en détail.

\subsection{Tests de performance entre les différents algorithmes}
\label{sec:tests}

\begin{question}
  Effectuer des tests pour mesurer le temps d'exécution de votre algorithme par
  rapport à la taille de l'entrée (nombre de sommets, nombre d'arcs, étiquettes
  sur les sommets).
\end{question}

\begin{question}
  Comparer les algorithmes implantés pour le calcul de chemins de type IV dans
  les question 4 et 5.
\end{question}

\begin{reponse}
  Pour le graphe \verb|/Repertoire_Graphes/exempleGraphe.txt|, un graphe de 10
  sommets et 15 arcs, les temps d'exécution pour trouver trois le chemin le plus
  court entre 3 paires de sommets choisis aléatoirement (pour lesquels il existe
  un chemin réalisable) sont donnés

\begin{verbatim}
Résultats de l'algorithme :
Chemin recherché                Temps d'exécution               Solution
Chemin de b à h                  0.000185                [('b', 1), ('b', 3), ('h', 4)]
Chemin de i à l                  0.000262                [('i', 1), ('i', 8), ('l', 9)]
Chemin de f à l                  0.000220                [('f', 1), ('f', 5), ('i', 6),
                                                          ('i', 8), ('l', 9)]


Résultats de l'optimisation :
Chemin recherché                Temps d'exécution               Solution
Chemin de b à h                  0.024072                [('Arc_b,h', 1.0)]
Chemin de i à l                  0.002960                [('Arc_i,l', 1.0)]
Chemin de f à l                  0.002942                [('Arc_f,i', 1.0),
                                                          ('Arc_i,l', 1.0)]
\end{verbatim}

  On remarque que l'algorithme de programmation linéaire est plus d'une ordre de
  magnitude plus lent. Cela n'est pas surprenant, vu que les programmes
  linéaires ne sont pas une bonne méthode pour résoudre des algorithmes dans les
  graphes (le nombre de contraintes croit exponentiellement).
\end{reponse}


%%% Local Variables:
%%% mode: latex
%%% TeX-master: "../main"
%%% End:

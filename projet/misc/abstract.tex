\begin{abstract}
  % In this report, the theoretical foundations of Algebraic Machine Learning,
  % introduced in \cite{martin-maroto_algebraic_2018} and
  % \cite{martin-maroto_finite_2021}, are studied, and a clear and complete
  % introduction to it is attempted. By
  This report presents a theoretical study of AML (Algebraic Machine Learning), a
  novel approach to Machine Learning introduced by Fernando Martin-Maroto and
  Gonzalo de Polavieja in \cite{martin-maroto_algebraic_2018} et
  \cite{martin-maroto_finite_2021}. Important concepts were singled out,
  implicit prerequisites were explicited, and a more formal connection with
  concepts in Universal Algebra was attempted. An implimentation of full
  crossing, an important operation for AML, was also built.
\end{abstract}

\renewcommand{\abstractname}{Résumé}
\begin{abstract}
  Ce rapport présente une étude théorique d'AML (Machine Learning Algébrique),
  une nouvelle approche au Machine Learning proposée par Fernando Martin-Maroto
  et Gonzalo de Polavieja dans \cite{martin-maroto_algebraic_2018} et
  \cite{martin-maroto_finite_2021}. Les concepts les plus importants ont été mis
  en avant, les prérequis implicites ont été explicités, et plusieurs concepts
  ont été reliés à l'Algèbre Universelle. Une implémentation du ``full
  crossing'', une opération importante pour l'AML, a été réalisée.
\end{abstract}
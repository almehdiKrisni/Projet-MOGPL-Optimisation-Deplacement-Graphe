%%%%%%%%%%%%%%%%%%
% General config %
%%%%%%%%%%%%%%%%%%

\documentclass{article}
\usepackage[margin=1.2in]{geometry}
\usepackage[english]{babel}

%%%%%%%%%%%%
% Packages %
%%%%%%%%%%%%

% Math packages
\usepackage{amsthm}
\usepackage{amsmath,amssymb}
\usepackage{amsfonts,mathtools}
% \usepackage{thmbox}
\usepackage{bm}
\usepackage{proof-at-the-end}

% graphics packages
\usepackage{pict2e,picture}
\usepackage{graphicx}
\usepackage{tikz}
\usetikzlibrary {positioning,graphs,calc,decorations.pathmorphing,shapes,arrows.meta,arrows,shapes.misc}
\usetikzlibrary{topaths,calc}
\usetikzlibrary{fit}
\usepackage{subcaption}
\usepackage{xcolor}
\usepackage{wrapfig}
\usepackage{float}

% custom environments
\usepackage{comment}
\usepackage{enumitem}
\usepackage{epigraph}
\usepackage{environ}
\usepackage{listings}

% Miscellanious packages
\usepackage{lipsum,mwe,abstract}
\usepackage{multicol}
\usepackage{refcount}
\usepackage{hyperref}
\usepackage{censor}
\usepackage[numbers]{natbib}



%%%%%%%%%%%%%%%%%%%
% Custom Commands %
%%%%%%%%%%%%%%%%%%%

% covering relation
\newcommand{\coveringA}{%
  \mathrel{-\mkern-4mu}<%
}
\newcommand{\coveringB}{\mathrel{\text{$\vcenter{\hbox{\pictcoveringB}}$}}}
\newcommand{\pictcoveringB}{%
  \begin{picture}(1em,.5em)
    \roundcap
    \put(0,.25em){\line(1,0){.6em}}
    \put(.6em,.25em){\line(3,1){.4em}}
    \put(.6em,.25em){\line(3,-1){.4em}}
  \end{picture}%
}

% highlight
\newcommand{\highlight}[1]{%
  \par\noindent
  \colorbox{gray!30}{%
    \parbox{\dimexpr\linewidth-2\fboxsep\relax}{%
      #1
    }%
  }}

% custom theorem environments
\theoremstyle{plain}
\newtheorem{theorem}{Theorem}[section]
\newtheorem{corollary}[theorem]{Corollary}
\newtheorem{lemma}[theorem]{Lemma}
\newtheorem{prop}[theorem]{Proposition}

\theoremstyle{definition}
\newtheorem{definition}[theorem]{Definition}
\newtheorem{example}[theorem]{Example}
\newtheorem{property}[theorem]{Property}
\newtheorem{notation}[theorem]{Notation}
\newtheorem{convention}[theorem]{Convention}
\newtheorem{interpretation}[theorem]{Interpretation}
\newtheorem{remark}[theorem]{Remark}
% \newtheorem{question}[theorem]{Question}
\newtheorem{note}[theorem]{Note}

\newtheorem{question}{Question}
\theoremstyle{plain}
\newtheorem{assertion}{Assertion}[question]

\newenvironment{reponse}{\renewcommand{\proofname}{Réponse}\begin{proof}}{\end{proof}}

% special theorems / definitions / lemmas
\newtheoremstyle{specialthm}% name
{\topsep}%   Space above
{\topsep}%   Space below
{\itshape}%  Body font
{}%          Indent amount
{\bfseries}% Theorem head font
{}%          Punctuation after theorem head -- blank
{0.5em}%     Space after theorem head (0.5em is the default)
{{\thmname{#1}\thmnumber{ #2$^{\bm*}\!$}\thmnote{\ \textmd{(#3)}.}}}

\theoremstyle{specialthm}
\newtheorem{sptheorem}[theorem]{Theorem}
\newtheorem{spcorollary}[theorem]{Corollary}
\newtheorem{splemma}[theorem]{Lemma}
\newtheorem{spprop}[theorem]{Proposition}

% special theorems / definitions / lemmas
\newtheoremstyle{specialdef}% name
{\topsep}%   Space above
{\topsep}%   Space below
{}%  Body font
{}%          Indent amount
{\bfseries}% Theorem head font
{}%          Punctuation after theorem head -- blank
{0.5em}%     Space after theorem head (0.5em is the default)
{{\thmname{#1}\thmnumber{ #2$^{\bm*}\!$}\thmnote{\ \textmd{(#3)}.}}}

\theoremstyle{specialdef}
\newtheorem{spdefinition}[theorem]{Definition}
\newtheorem{spproperty}[theorem]{Property}

% abstract customization
\renewenvironment{abstract}
{\small
  \begin{center}
    \bfseries \abstractname\vspace{-.5em}\vspace{0pt}
  \end{center}
  \list{}{
    \setlength{\leftmargin}{20mm}
    \setlength{\rightmargin}{\leftmargin}
  }
\item\relax}
{\endlist}

% special epigraph/acknowledgements
\newenvironment{dedication}
{\clearpage           % we want a new page
  \vspace*{\stretch{1}}% some space at the top
  \itshape             % the text is in italics
  \raggedleft          % flush to the right margin
  \begin{minipage}{0.6\linewidth}
    \parindent=12pt
  }
  {\end{minipage}
  \par % end the paragraph
  \vspace{\stretch{2}} % space at bottom is three times that at the top
  \raggedleft
  \textit{All Finite Lattices are Algebraic Lattices.}\\
  \textit{-- G. Birkhoff}
  \vspace{\stretch{1}}
  \clearpage           % finish off the page
}

% Size for partition
% \def\Bign#1{\mathclose{\hbox{$\left#1\vbox to11.5\p@{}\right.\n@space$}}\mathopen{}}

% equiv class
\def\equivclass{/}

\label{proofsection:prAtEndii}\begin{proof}[Proof of \autoref{thm:prAtEndii}]\phantomsection\label{proof:prAtEndii}Let's show that $a \mathrel {\theta } b \implies [a] = [b]$ if $\theta $ is an equivalence relation. \par Let's suppose $a \mathrel {\theta } b$. We will show $[a] \subseteq [b]$ and $[b] \subseteq [a]$. \par Let $x \in [a]$, so $x \mathrel {\theta } a$. We have $x \mathrel {\theta } a$ and $a \mathrel {\theta } b$, so by transitivity of the equivalence relation, $x \mathrel {\theta } b \implies x \in [b]$. We have shown $x \in [a] \implies x \in [b]$, so $[a] \subseteq [b]$. \par Let $x \in [b]$, so $x \mathrel {\theta } b$. We have $a \mathrel {\theta } b \implies b \mathrel {\theta } a$, by symmetry of the equivalence relation, so $x \mathrel {\theta } b$ and $b \mathrel {\theta } a$ implies $x \mathrel {\theta } a$ proving $x \in [a]$. Thus $[b] \subseteq [a]$.\end{proof}
\label{proofsection:prAtEndiii}\begin{proof}[Proof of \autoref{thm:prAtEndiii}]\phantomsection\label{proof:prAtEndiii}Trivial, using Proposition \ref {prop:equiv_class_equal}.\end{proof}
\label{proofsection:prAtEndiv}\begin{proof}[Proof of \autoref{thm:prAtEndiv}]\phantomsection\label{proof:prAtEndiv}Let $A_1, ...,A_n$ denote the equivalence classes of a nonempty set $A$. We must prove that $\bigcup A_i = A$ and that $\bigcap A_i = \emptyset $ in order for $A_1, ...,A_n$ to be a partition of $A$. \par Let's show that $\bigcup A_i = A$: we will show $\bigcup A_i \subseteq A$ and $A \subseteq \bigcup A_i$. \par Let $x \in \bigcup A_i$, so $\exists i$ such that $x \in A_i$, by definition of the union. Since $A_i = \{x \in A | x \mathrel {\theta } a_i\}$, where $[a_i] = A_i$, we have $x \in A_i \implies x \in A$. \par Now let $x \in A$. We have $x \in [x]$, by reflexivity of the equivalence relation, and by definition of the set of equivalence classes $\exists i$ such that $[x] = A_i$. Therefore, $x \in \bigcup A_i$ since $[x] = A_i \iff x \in A_i$. \par Now let's show that $\bigcap A_i = \emptyset $. \par By definition of the set of equivalence classes, $A_i \neq A_j$, for $i \neq j$ given that they are disjoint. Therefore, using Lemma \ref {lem:equiv_rel_eq}, $A_i \cap A_j = \emptyset $, for $i \neq j$. This proves that $\bigcap A_i = \emptyset $.\end{proof}
\label{proofsection:prAtEndv}\begin{proof}[Proof of \autoref{thm:prAtEndv}]\phantomsection\label{proof:prAtEndv}For $\leq $ to be a partial order it must be reflexive, anti-symmetric and transitive. Let $x, y, z \in M$: \begin {enumerate}[itemsep=2pt, topsep=2pt,parsep=0pt,partopsep=0pt] \item (reflexive: $x \leq x$) $x \odot x = x$ by indempotence. \item (anti-symmetric: $x \leq y$ and $y \leq x$ implies $x = y$) Suppose $x \odot y = y$ and $y \odot x = x$. By commutativity, $x \odot y = y \odot x$, so $x = y$. \item (transitivity: if $x \leq y$ and $y \leq z$ then $x \leq z$) Suppose $x \odot y = y$ and $y \odot z = z$. By associativity, $y \odot z = (x \odot y) \odot z = x \odot (y \odot z) = x \odot z = z$. \end {enumerate} Thus $\leq $ is a partial order.\end{proof}
\label{proofsection:prAtEndvi}\begin{proof}[Proof of \autoref{thm:prAtEndvi}]\phantomsection\label{proof:prAtEndvi}Let $\theta \in \mathrm {Con}(\boldsymbol {A})$ and consider the natural map $\nu _{\theta } : A \rightarrow A / \theta $ that maps elements of the algebra $\boldsymbol {A}$ to the quotient of that algebra $\boldsymbol {A} / \theta $. For any $n$-ary function $f \in \mathcal {F}$ and elements $a_1, \dots , a_n \in A$, we have: $\nu _{\theta }(f^{\boldsymbol {A}}(a_1, \dots , a_n)) = f^{\boldsymbol {A}}(a_1, \dots , a_n) / \theta $. Since $\nu _{\theta }$ maps to a quotient of the algebra, by Equation \ref {eq:quotient_algebra} of Definition \ref {def:quotient_algebra}, $\nu _{\theta }(f^{\boldsymbol {A}}(a_1, \dots , a_n)) = f^{\boldsymbol {A} / \theta }(a_1 / \theta , \dots , a_n / \theta )$. Since the natural map is defined by $\nu _{\theta }(a) = a / \theta $ (Definition \ref {def:natural_map}), we have $\nu _{\theta }(f^{\boldsymbol {A}}(a_1, \dots , a_n)) = f^{\boldsymbol {A} / \theta }(\nu _{\theta }(a_1), \dots , \nu _{\theta }(a_n))$. As such, by Equation \ref {eq:morphism} of Definition \ref {def:homomorphism}, $\nu _{\theta }$ is a homomorphism. \par Since $A / \theta $ is the quotient set of $A$, by Theorem \ref {the:partition} given Notation \ref {not:quotient_set}, $A / \theta $ forms a partition of $A$. And since $A / \theta = \{\{b \in A \mid a \mathrel {\theta } b\} \mid a \in A\}$, for every $x \in A / \theta $ there exists $y \in A$ such that $[y]_{\theta } = x$. Thus $\nu _{\theta }$ is an onto homomorphism.\end{proof}
\label{proofsection:prAtEndvii}\begin{proof}[Proof of \autoref{thm:prAtEndvii}]\phantomsection\label{proof:prAtEndvii}Consider two algebras $\boldsymbol {A}$ and $\boldsymbol {B}$ of the same type $\mathcal {F}$, with $f$ being an $n$-ary function in $\mathcal {F}$, and consider an homomorphism $\alpha : \boldsymbol {A} \rightarrow \boldsymbol {B}$. Let $\langle a_i,b_i \rangle \in \ker (\alpha )$. \par Since $\ker (\alpha )$ is defined by $\alpha (a) = \alpha (b)$ using the equivalence relation ``$=$'' which is reflexive, symmetric and transitive, $\ker (\alpha )$ is an equivalence relation. \par We have: $\alpha (f^{\boldsymbol {A}}(a_1, \dots , a_n)) = f^{\boldsymbol {B}}(\alpha (a_1), \dots , \alpha (a_n))$. \par Since $\langle a_i,b_i \rangle \in \ker (\alpha )$, $\alpha (f^{\boldsymbol {A}}(a_1, \dots , a_n)) = f^{\boldsymbol {B}}(\alpha (b_1), \dots , \alpha (b_n))$, and by Equation \ref {eq:morphism} of Definition \ref {def:homomorphism} (since $\alpha $ is a homomorphism), $\alpha (f^{\boldsymbol {A}}(a_1, \dots , a_n)) = \alpha (f^{\boldsymbol {B}}(b_1, \dots , b_n))$. \par As such, $\langle \alpha (f^{\boldsymbol {A}}(a_1, \dots , a_n)), \alpha (f^{\boldsymbol {B}}(b_1, \dots , b_n)) \rangle \in \ker (\alpha )$. \par The equivalence relation $\ker (\alpha )$ verifies Property \ref {prop:compatibility}, thus is compatible with the operations of $\boldsymbol {A}$. As such, $\ker (\alpha )$ is a congruence on $\boldsymbol {A}$.\end{proof}
\label{proofsection:prAtEndviii}\begin{proof}[Proof of \autoref{thm:prAtEndviii}]\phantomsection\label{proof:prAtEndviii}Given that $|\mathcal {P}(C)| = 2^{|C|}$, and that in $F_C$ we don't consider the empty set in $\mathcal {P}(C)$, we have $|F_C(\emptyset )| = 2^{|C|} -1$.\end{proof}
\label{proofsection:prAtEndix}\begin{proof}[Proof of \autoref{thm:prAtEndix}]\phantomsection\label{proof:prAtEndix}We know from definition \ref {def:freealgebra} that $\boldsymbol {F}_C(\emptyset ) = \boldsymbol {T}(\emptyset ) / \theta _C(\emptyset )$ where $\theta _C(\emptyset )$ is the congruence defined by Equation \ref {eq:smallestcong}. We know from Theorem \ref {the:homothe} that there is always a natural homomorphism from $\boldsymbol {T}(\emptyset )$ onto $\boldsymbol {T}(\emptyset ) / \theta _C(\emptyset )$, the free algebra. Since all our sets are finite, we can simply prove Equation \ref {eq:freeisterm} by proving $\left | T(\emptyset ) \right | = \left | T(\emptyset ) / \theta _C(\emptyset ) \right |$, and we know that in order for the cardinalities to be the same, $\theta _C(\emptyset )$ must be the \textit {diagonal relation} $\Delta = \{\langle t,t \rangle : t \in T(\emptyset )\}$. \par Let $\psi \in \text {Con}(\boldsymbol {T}(\emptyset ))$, by Definition \ref {def:term_algebra} it is clear that no elements of the term algebra are equivalent (since each element is a unique summation of constants) so $\psi = \Delta $ and it follows that Equation \ref {eq:smallestcong} becomes $\theta _C(\emptyset ) = \bigcap \Phi _C(\emptyset ) = \Delta $ since $\Phi _C(\emptyset ) \subseteq \{ \psi \in \text {Con}(\boldsymbol {T}(\emptyset ))\}$.\end{proof}
\label{proofsection:prAtEndx}\begin{proof}[Proof of \autoref{thm:prAtEndx}]\phantomsection\label{proof:prAtEndx}Since $X = \emptyset $, the conditions for the existence of a function $f : \emptyset \rightarrow M$ are vacuously true. Since the term algebra has the universal mapping proprety (Theorem \ref {the:term_a_ump}), we know there exists an homomorphism $\nu _{\boldsymbol {M}}$ between $\boldsymbol {T}(\emptyset )$ and $\boldsymbol {M}$. \par In order to prove $\nu _{\boldsymbol {M}}$ is onto, we will use a decomposition similar to the one of Theorem \ref {the:homothe}. First, we define the kernel of $\nu _{\boldsymbol {M}}$ which we know, from \ref {the:keriscong}, is a congruence on $\boldsymbol {T}(\emptyset )$. This defines a natural homomorphism $\nu : \boldsymbol {T}(\emptyset ) \rightarrow \boldsymbol {T}(\emptyset )/\ker (\nu _{\boldsymbol {M}})$ which is onto by Theorem \ref {the:naturalyonto}. Using Theorem \ref {the:partition} we know that the universe of the inverse image of the natural homomorphism partitions the elements of $T(\emptyset )$ into equivalence classes and since $\ker (\nu _{\boldsymbol {M}}) = \{ \langle a,b\rangle \in T(\emptyset ) : \nu _{\boldsymbol {M}}(a) = \nu _{\boldsymbol {M}}(b) \}$, there are as many equivalence classes (ie. elements of the partition) as there are elements in $M$. It folows that $|T(\emptyset )/\ker (\nu _{\boldsymbol {M}})| = |M|$. Therefore, there is an isomorphism $\phi $ from $\boldsymbol {T}(\emptyset )/\ker (\nu _{\boldsymbol {M}})$ to $\boldsymbol {M}$ (since they share the same cardinality) such that $\nu _{\boldsymbol {M}} = \phi \circ \nu $; proving $\nu _{\boldsymbol {M}}$ is onto by the composition of a bijection with a surjection.\end{proof}
\label{proofsection:prAtEndxi}\begin{proof}[Proof of \autoref{thm:prAtEndxi}]\phantomsection\label{proof:prAtEndxi}This is a direct application of the following logical formula: $ (A \lor B \iff B) \iff (A \implies B) $\end{proof}
\label{proofsection:prAtEndxii}\begin{proof}[Proof of \autoref{thm:prAtEndxii}]\phantomsection\label{proof:prAtEndxii}Let's suppose (AS3) and $a \leq b$. Then by AS3: $\phi < a \Rightarrow \phi < b$, which by Lemma \ref {lemma:tautology}, is equivalent to $\phi < a \lor \phi < b \iff \phi < b$. Axiom AS4, given the last result, becomes: $\phi < a \odot b \iff \phi < b$. This equivalence gives us two implications: $\phi < a \odot b \Rightarrow \phi < b$ and $\phi < b \Rightarrow \phi < a \odot b$. By applying AS3 (from right to left) to both we get: $a \odot b \leq b$ and $b \leq a \odot b$. By anti-symmetry: $a \odot b = b$. \par Now suppose $a \odot b = b$. By AS4 we get $\phi < b \iff \phi < a \lor \phi < b$, which by Lemma \ref {lemma:tautology} immediatly gives us $\phi < a \Rightarrow \phi < b$.\end{proof}
\label{proofsection:prAtEndxiii}\begin{proof}[Proof of \autoref{thm:prAtEndxiii}]\phantomsection\label{proof:prAtEndxiii}Suppose (AS3b) and $a \leq b$. By (AS3b) we get $a \odot b = b$ so (AS4) becomes $\phi < b \iff \phi < a \lor \phi < b$. By Lemma \ref {lemma:tautology} we get $\phi < a \Rightarrow \phi < b$. \par Now suppose $\phi < a \Rightarrow \phi < b$. We want to prove $a \leq b$ which is equivalent to proving $a \odot b = b$ given (AS3b). By the lemma 1.1, the hypothesis $\phi < a \Rightarrow \phi < b$ gives us $\phi < a \lor \phi < b \iff \phi < b$. Now AS4 gives us $\phi < a \odot b \iff \phi < b$ so, by applying the hypothesis of the lemma (ie. $((\phi < a) \iff (\phi < b)) \implies (a = b)$) to $a \odot b$ and $b$, we get $a \odot b = b$.\end{proof}
\label{proofsection:prAtEndxiv}\begin{proof}[Proof of \autoref{thm:prAtEndxiv}]\phantomsection\label{proof:prAtEndxiv}Let's suppose $(\phi < a \iff \phi < b)$; which gives us two implications: $(\phi < a \Rightarrow \phi < b)$ and $(\phi < b \Rightarrow \phi < a)$. Given (AS3) (ie. $a \leq b \iff (\phi < a \Rightarrow \phi < b)$) we get, from the two implications, $a \leq b$ and $b \leq a$. Finaly, by anti-symmetry we have $a=b$.\end{proof}
\label{proofsection:prAtEndxv}\begin{proof}[Proof of \autoref{thm:prAtEndxv}]\phantomsection\label{proof:prAtEndxv}We will refer to $\forall a,b, \forall \phi , ((\phi < a) \iff (\phi < b)) \implies (a = b)$ as $\star $. \par From Lemma \ref {lemma:1.2}, (AS3) $\implies $ (AS3b), and from Lemma \ref {lemma:1.4}, (AS3) implies $\star $. \par Let's prove (AS3b) implies (AS3). Suppose we have (AS3b). Given $\star $, we have (AS3b) $\implies $ (AS3), through Lemma \ref {lemma:1.3}. \par Since (AS3b) implies (AS3), by transitivity, (AS3b) implies $\star $.\end{proof}
\label{proofsection:prAtEndxvi}\begin{proof}[Proof of \autoref{thm:prAtEndxvi}]\phantomsection\label{proof:prAtEndxvi}\begin {enumerate}[label=\roman *)] \item Let $c \in \boldsymbol {C}(t)$. By definition, $c \leq t$. Using (AS3b) we have $t = c \odot t$. Then $\nu _{\boldsymbol {M}}(t) = \nu _{\boldsymbol {M}}(c \odot t)$ and by the proprety of homomorphisms $\nu _{\boldsymbol {M}}(c \odot t) = \nu _{\boldsymbol {M}}(c) \odot \nu _{\boldsymbol {M}}(t)$. Thefore, $\nu _{\boldsymbol {M}}(t) = \nu _{\boldsymbol {M}}(t) \odot \nu _{\boldsymbol {M}}(c)$ which by (AS3b) gives us $\nu _{\boldsymbol {M}}(c) \leq \nu _{\boldsymbol {M}}(t)$. \item Suppose $\phi < \nu _{\boldsymbol {M}}(t)$. We know that any regular element is either a constant or a merge of constants so we can write $\nu _{\boldsymbol {M}}(t) = \bigodot _{i \geq 1} c_i$ (if $i = 1$, then $\nu _{\boldsymbol {M}}(t)$ is a constant). Now, we will use (AS4) generalized to a merge of a family of terms $(t_i)_{i \geq 1}$: \begin {equation} \label {eq:extandedAS4} \phi < \bigodot _{i \geq 1} t_i \iff \bigvee _{i \geq 1} \phi < t_i \end {equation} Equation \ref {eq:extandedAS4} applied to our term $\nu _{\boldsymbol {M}}(t)$ give us: $\phi < \nu _{\boldsymbol {M}}(t) \iff \bigvee _{i \geq 1} \phi < c_i$. Given that $\phi < \nu _{\boldsymbol {M}}(t)$, the right side of the previous equation must hold, meaning there exists at least one $c$ that verifies $\phi < c$. Then $c \in \boldsymbol {C}(t)$ by definition. We proved that $\exists c: ((c \in \boldsymbol {C}(t)) \land (\phi < \nu _{\boldsymbol {M}}(c)))$. \par Now let's suppose $\exists c: ((c \in \boldsymbol {C}(t)) \land (\phi < \nu _{\boldsymbol {M}}(c)))$. Given i) we know that $\nu _{\boldsymbol {M}}(c) \leq \nu _{\boldsymbol {M}}(t) $ which by transitivity proves that $\phi < \nu _{\boldsymbol {M}}(t)$. \par \item Suppose $\phi < a$. Let $t \in \boldsymbol {F_C}(\emptyset )$ such that $a = \nu _{\boldsymbol {M}}(t)$. We can then use proposition ii) to prove that $\exists c: (c \in C) \land (\phi < \nu _{\boldsymbol {M}}(c))$. From proposition i we have $\nu _{\boldsymbol {M}}(c) \leq a$, therefore we have proven $\exists c: (c\in C) \land (\phi < \nu _{\boldsymbol {M}}(c) \leq a)$. Now, suppose $\exists c: (c\in C) \land (\phi < \nu _{\boldsymbol {M}}(c))$. We have $\phi < \nu _{\boldsymbol {M}}(c) \leq a$ which gives us $\phi < a$. \end {enumerate}\end{proof}
\label{proofsection:prAtEndxvii}\begin{proof}[Proof of \autoref{thm:prAtEndxvii}]\phantomsection\label{proof:prAtEndxvii}\begin {enumerate}[label=\roman *)] \item Let $\phi \in \boldsymbol {L}_{\boldsymbol {M}}^a(\nu _{\boldsymbol {M}}(t))$, then $\phi < \nu _{\boldsymbol {M}}(t)$. By ii, $\exists c: (c\in \boldsymbol {C}(t)) \land (\phi < \nu _{\boldsymbol {M}}(c))$. Therefore, $c \in \boldsymbol {C}(t)$ and $c \in U^c_{\boldsymbol {M}}(\phi )$ (given that $U^c_{\boldsymbol {M}}$ is the set of constants $c \in C$ that satisfy $\phi < c$ in the model $\boldsymbol {M}$). \par Now let $\psi \in \{\phi \in M : \boldsymbol {C}(t) \cap \boldsymbol {U}^c(\phi ) \neq \emptyset \}$. By definition, there exists a term $t$ such that $c \in \boldsymbol {C}(t) \cap \boldsymbol {U}^c_{\boldsymbol {M}}(\psi )$, so $c \leq t$ and $\psi < c$ which gives us $\psi < t$. Therefore $\psi \in \boldsymbol {L}^a_{\boldsymbol {M}}(\nu _{\boldsymbol {M}}(t) )$. \item Let $\psi \in \boldsymbol {L}^a_{\boldsymbol {M}}(\nu _{\boldsymbol {M}}(s) \odot \nu _{\boldsymbol {M}}(t))$. Then $\psi \in \boldsymbol {L}^a_{\boldsymbol {M}}(\nu _{\boldsymbol {M}}(s \odot t))$. So $\psi < s \odot t$, and by (AS4) we have $\psi < s \lor \psi < t$. Therefore, $\psi \in \boldsymbol {L}^a_{\boldsymbol {M}}(\nu _{\boldsymbol {M}}(t)) \cup \boldsymbol {L}^a_{\boldsymbol {M}}(\nu _{\boldsymbol {M}}(s))$. \par Let $\psi \in \boldsymbol {L}^a_{\boldsymbol {M}}(\nu _{\boldsymbol {M}}(t)) \cup \boldsymbol {L}^a_{\boldsymbol {M}}(\nu _{\boldsymbol {M}}(s))$. We have $\psi < t \lor \psi < s$. So $\psi < s \odot t$, therefore $\psi \in \boldsymbol {L}^a_{\boldsymbol {M}}(\nu _{\boldsymbol {M}}(s \odot t))$ and $\psi \in \boldsymbol {L}^a_{\boldsymbol {M}}(\nu _{\boldsymbol {M}}(s) \odot \nu _{\boldsymbol {M}}(t))$. \item From (AS3b), we know that $\nu _{\boldsymbol {M}}(t) \leq \nu _{\boldsymbol {M}}(s) \iff \nu _{\boldsymbol {M}}(t) \odot \nu _{\boldsymbol {M}}(s) = \nu _{\boldsymbol {M}}(s)$. Using proprety v) we get $\nu _{\boldsymbol {M}}(t) \leq \nu _{\boldsymbol {M}}(s) \iff \boldsymbol {L}_{\boldsymbol {M}}^{a}(\nu _{\boldsymbol {M}}(t)) \cup \boldsymbol {L}_{\boldsymbol {M}}^{a}(\nu _{\boldsymbol {M}}(s)) = \boldsymbol {L}_{\boldsymbol {M}}^{a}(\nu _{\boldsymbol {M}}(s))$ which gives us $\nu _{\boldsymbol {M}}(t) \leq \nu _{\boldsymbol {M}}(t) \iff \boldsymbol {L}_{\boldsymbol {M}}^{a}(\nu _{\boldsymbol {M}}(t)) \subseteq \boldsymbol {L}_{\boldsymbol {M}}^{a}(\nu _{\boldsymbol {M}}(s))$ given that for any sets $A$ and $B$: $A \cup B = B \iff A \subseteq B$. \end {enumerate}\end{proof}
\label{proofsection:prAtEndxviii}\begin{proof}[Proof of \autoref{thm:prAtEndxviii}]\phantomsection\label{proof:prAtEndxviii}Immediate, by using proposition iii) of Theorem \ref {thm:asl2} and axiom (AS3b).\end{proof}
\label{proofsection:prAtEndxix}\begin{proof}[Proof of \autoref{thm:prAtEndxix}]\phantomsection\label{proof:prAtEndxix}\begin {enumerate}[label=\roman *), align=left, itemsep=2pt, topsep=2pt,parsep=0pt,partopsep=0pt] \item Suppose $\boldsymbol {F_C}(\emptyset ) \models (t \leq s)$. By (AS3b), $\boldsymbol {F_C}(\emptyset ) \models t \odot s = s$. Applying $\nu _{\boldsymbol {M}}$ gives us $\boldsymbol {M} \models \nu _{\boldsymbol {M}}(t \odot s) = \nu _{\boldsymbol {M}}(s)$. By the properties of the natural homomorphism, $\boldsymbol {M} \models \nu _{\boldsymbol {M}}(t) \odot \nu _{\boldsymbol {M}}(s) = \nu _{\boldsymbol {M}}(s)$. By (AS3b), $\boldsymbol {M} \models \nu _{\boldsymbol {M}}(t) \leq \nu _{\boldsymbol {M}}(s)$. \par \item First, we prove the statement from left to right. \par Let $(\phi \in \boldsymbol {M})$. By Definition 1.14, $\forall c \in \boldsymbol {U}^{c}(\phi )$, $\phi < \nu _{\boldsymbol {M}}(c) \in \boldsymbol {M}$. \par Let $\boldsymbol {F_C}(\emptyset ) \models (\phi < s)$. As such, $\exists c \in \boldsymbol {C}(s)$, such that $\phi < c$ and $c \leq s$. Since $\phi < c$, $c \in \boldsymbol {U}^{c}(\phi )$. As seen previously, if $c \in \boldsymbol {U}^{c}(\phi )$, then $\phi < \nu _{\boldsymbol {M}}(c)$. \par Using the first property of this theorem, $c \leq s$ implies $\nu _{\boldsymbol {M}}(c) \leq \nu _{\boldsymbol {M}}(s)$. And since $\phi < \nu _{\boldsymbol {M}}(c)$, by transitivity of the order relation, we have $\phi < \nu _{\boldsymbol {M}}(s)$. \par It remains to prove that the statement holds from right to left. \par Let $\boldsymbol {M} \models \phi < \nu _{\boldsymbol {M}}(s)$. It is trivial that $\phi \in M$. Let's now show that $\boldsymbol {F_C}(\emptyset ) \models (\phi < s)$ follows from our hypothesis. \par Assume $\boldsymbol {M} \models (\phi < \nu _{\boldsymbol {M}})$. By Theorem \ref {thm:asl1} proposition ii), there exists $c \in \boldsymbol {C}(s)$ such that $\phi < \nu _{\boldsymbol {M}}(c)$. \par From this, a non-trivial deduction can be made: $c \in \boldsymbol {U}^{c}(\phi )$. Assume otherwise, that $c \notin \boldsymbol {U}^{c}(\phi )$. Since $\phi < \nu _{\boldsymbol {M}}(c)$, $\exists c^{'} \in C$ such that $c^{'} \in \boldsymbol {U}^{c}(\phi )$ and $\nu _{\boldsymbol {M}}(c^{'}) \leq \nu _{\boldsymbol {M}}(c)$. But then, $\phi $ would discriminate the duple $(\nu _{\boldsymbol {M}}(c^{'}),\nu _{\boldsymbol {M}}(c))$, which is a contradiction. As such, $c \in \boldsymbol {U}^{c}(\phi )$. \par So $\phi < \nu _{\boldsymbol {M}}(c)$ implies that $\phi \in \boldsymbol {L}^{a}_{\boldsymbol {M}}(\nu _{\boldsymbol {M}}(t))$, which by Theorem \ref {thm:asl2} proposition ii) implies that $\boldsymbol {C}(c) \cap \boldsymbol {U}^{c}(\phi ) \neq \emptyset $. But $\boldsymbol {C}(c) = \{c\}$, so $c \in \boldsymbol {U}^{c}(\phi )$. \par Since $c \in \boldsymbol {C}(s)$, it is seen that $\phi < c \leq s$. The result follows. \end {enumerate}\end{proof}
